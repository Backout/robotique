\documentclass[]{article}
\usepackage{amssymb,amsmath}
\usepackage{ifxetex,ifluatex}
\ifxetex
  \usepackage{fontspec,xltxtra,xunicode}
  \defaultfontfeatures{Mapping=tex-text,Scale=MatchLowercase}
\else
  \ifluatex
    \usepackage{fontspec}
    \defaultfontfeatures{Mapping=tex-text,Scale=MatchLowercase}
  \else
    \usepackage[utf8]{inputenc}
  \fi
\fi
\usepackage{listings}
\ifxetex
  \usepackage[setpagesize=false, % page size defined by xetex
              unicode=false, % unicode breaks when used with xetex
              xetex,
              colorlinks=true,
              linkcolor=blue]{hyperref}
\else
  \usepackage[unicode=true,
              colorlinks=true,
              linkcolor=blue]{hyperref}
\fi
\hypersetup{breaklinks=true, pdfborder={0 0 0}}
\newcommand{\textsubscr}[1]{\ensuremath{_{\scriptsize\textrm{#1}}}}
\setlength{\parindent}{0pt}
\setlength{\parskip}{6pt plus 2pt minus 1pt}
\setlength{\emergencystretch}{3em}  % prevent overfull lines

\title{TD1. Étude du robot Scara}
\author{Axel BENEULT LEFAUCHEUX}
\date{\today}

\begin{document}

\maketitle

\section{Établir le modèle géométrique direct en position et en
rotation.}

On utilise la table de Denavit et Hertenberg pour définir les matrices :
$H_1$, $H_2$, $H_3$, $H_4$.

\[ 
  H_1 =
  \begin{pmatrix}
     \cos (\theta^*_1) & -\sin (\theta^*_1) & 0 & a_1 \cos (\theta^*_1) \\
     \sin (\theta^*_1) & \cos (\theta^*_1) & 0 & a_1 \sin (\theta^*_1) \\
     0 & 0 & 1 & d_1 \\
     0 & 0 & 0 & 1 \\
  \end{pmatrix}
\]

\[ 
  H_2 =
  \begin{pmatrix}
    \cos (\theta^*_2) & \sin (\theta^*_2) & 0 & a_2 \cos (\theta^*_2) \\
    \sin (\theta^*_2) & -\cos (\theta^*_2) & 0 & a_2 \sin (\theta^*_2) \\
    0 & 0 & -1 & 0 \\
    0 & 0 & 0 & 1 \\
  \end{pmatrix}
\]

\[ 
  H_3 =
  \begin{pmatrix}
    1 & 0 & 0 & 0 \\
    0 & 1 & 0 & 0 \\
    0 & 0 & 1 & d^*_3 \\
    0 & 0 & 0 & 1 \\
  \end{pmatrix}
\]

\[ 
  H_4 =
  \begin{pmatrix}
    \cos (\theta^*_4) & -\sin (\theta^*_4) & 0 & 0 \\
    \sin (\theta^*_4) & \cos (\theta^*_4) & 0 & 0 \\
    0 & 0 & 1 & d_4 \\
    0 & 0 & 0 & 1 \\
  \end{pmatrix}
\]

En calculant le produit matricielle de ces matrices on obtient une
matrice contenant le modèle géométrique direct.

\[
  H_0^4 = H_1.H_2.H_3.H_4 
\]

\[ 
  H_0^4 =
  \begin{pmatrix}
    \cos (\theta^*_1+\theta^*_2-\theta^*_4) & \sin (\theta^*_1+\theta^*_2-\theta^*_4) & 0 & a_1 \cos (\theta^*_1)+a_2 \cos (\theta^*_1+\theta^*_2) \\
    \sin (\theta^*_1+\theta^*_2-\theta^*_4) & -\cos (\theta^*_1+\theta^*_2-\theta^*_4) & 0 & a_1 \sin (\theta^*_1)+a_2 \sin (\theta^*_1+\theta^*_2) \\
    0 & 0 & -1 & d_1-d^*_3-d_4 \\
    0 & 0 & 0 & 1 \\
  \end{pmatrix}
\]

A partir de la matrice $H_0^4$ on peut
déduire le modèle géométrique direct en position (MGDpos) et en
orientation (MGDori)

\[ 
  MGDpos =
  \begin{pmatrix}
    a_1 \cos (\theta^*_1)+a_2 \cos(\theta^*_1+\theta^*_2) \\
    a_1 \sin (\theta^*_1)+a_2 \sin(\theta^*_1+\theta^*_2) \\
    d_1-d^*_3-d_4 \\
  \end{pmatrix}
\]

\[ 
  MGDori =
  \begin{pmatrix}
    \cos (\theta^*_1+\theta^*_2-\theta^*_4) & \sin (\theta^*_1+\theta^*_2-\theta^*_4) & 0 \\
    \sin (\theta^*_1+\theta^*_2-\theta^*_4) & -\cos (\theta^*_1+\theta^*_2-\theta^*_4) & 0 \\
    0 & 0 & -1\\
  \end{pmatrix}
\]

\section{Inverser le modèle géométrique en position.}

Pour inverser le modèle géométrique en position on utilise une méthode
analytique. Depuis le MGD en position on peut en déduire que :

\[
  \begin{pmatrix}
    X\\
    Y\\
    Z\\
  \end{pmatrix}
  = f
  \begin{pmatrix}
    \theta^*_1 \\
    \theta^*_2 \\
    d^*_3 \\
  \end{pmatrix}
\]

\[
  \begin{pmatrix}
    X\\
    Y\\
    Z\\
  \end{pmatrix}
  = 
  \begin{pmatrix}
    a_1 \cos (\theta^*_1)+a_2 \cos(\theta^*_1+\theta^*_2) \\
    a_1 \sin (\theta^*_1)+a_2 \sin(\theta^*_1+\theta^*_2) \\
    d_1-d^*_3-d_4 \\
  \end{pmatrix}
\]

Cherchant $\theta^*_1$, $\theta^*_2$ et $d^*_3$ il suffit de trouver
$f^{-1}$.

\begin{equation}
  Z = d_1-d^*_3-d_4
  \label{zeqd}
\end{equation}

L'équation \eqref{zeqd} nous permet de déduire directement que : 
\[
  d^*_3 = d_1-d_4-Z
\]

Pour exprimer $\theta^*_2$ on prend la somme de $X^2$ et
$Y^2$ pour pouvoir faire sortir un $\cos(\theta^*_2)$ :

\[
  X^2 + Y^2 ⁼ a^2_1 + a^2_2 + 2 a_1 a_2 \cos(\theta^*_2) 
\]

\[ 
  \cos(\theta^*_2) = \frac{X^2+Y^2-a_1^2-a_2^2}{2 a_1 a_2} 
\]

On peut donc dire que si :
$| \frac{X^2+Y^2-a_1^2-a_2^2}{2 a_1 a_2} | \leq 1$ alors
$\theta^*_2 = \pm \arccos(\frac{X^2+Y^2-a_1^2-a_2^2}{2 a_1 a_2})$. 

Pour déterminer $\theta^*_1$, on considère $\theta^*_2$ comme connu.

\[ 
  X = a_1\cos(\theta^*_1)+a_2\cos(\theta^*_2)\cos(\theta^*_1)-a_2\sin(\theta^*_2)\sin(\theta^*_1)
\]
\[
  Y = a_1\sin(\theta^*_1)+a_2\cos(\theta^*_2)\cos(\theta^*_1)+a_2\sin(\theta^*_1)\cos(\theta^*_2)
\]

\[ 
  X = \underbrace{a_1\cos(\theta^*_1)+a_2\cos(\theta^*_2)}_{k_1}\cos(\theta^*_1)-\underbrace{a_2\sin(\theta^*_2)}_{k_2}\sin(\theta^*_1)
\]
\[
  Y = \underbrace{a_1\cos(\theta^*_1)+a_2\cos(\theta^*_2)}_{k_1}\sin(\theta^*_1)+\underbrace{a_2\sin(\theta^*_2)}_{k_2}\cos(\theta^*_1)
\]

\[ 
  X = k_1\cos(\theta^*_1)-k_2\sin(\theta^*_1)
\]
\[ 
  Y = k_2\cos(\theta^*_1)+k_1\sin(\theta^*_1)
\]

Si on pose $r = \sqrt{k_1^2+k_2^2}$ on peut écrire 
\[
\begin{array}{lcl} 
  \frac{x}{r} & = & \cos(\alpha)\cos(\theta_1)-\sin(\alpha)\sin(\theta_1)\\
  & = & \cos(\theta_1+\alpha)
\end{array}
\]

\[
\begin{array}{lcl} 
  \frac{y}{r} & = & \sin(\alpha)\cos(\theta_1)+\cos(\alpha)\sin(\theta_1)\\
  & = & \sin(\theta_1+\alpha)
\end{array}
\]

avec $\alpha = \mathrm{atan2}(k_2,k_1)$

\[
\begin{array}{lcl} 
  \Rightarrow \theta_1 + \alpha & = & \mathrm{atan2}(y,x)\\
  \Rightarrow \theta_1 & = & \mathrm{atan2}(y,x) - \mathrm{atan2}(k_2,k_1)
\end{array}
\]


Pour $\theta_4$
\[ 
  MGDori =
  \begin{pmatrix}
    \cos (\theta^*_1+\theta^*_2-\theta^*_4) & \sin (\theta^*_1+\theta^*_2-\theta^*_4) & 0 \\
    \sin (\theta^*_1+\theta^*_2-\theta^*_4) & -\cos (\theta^*_1+\theta^*_2-\theta^*_4) & 0 \\
    0 & 0 & -1\\
  \end{pmatrix}
\]

\subsection{Ce modèle admet-il une solution unique?}

Non, ce modèle admet deux solutions à cause de $\theta_2$ qui est égal à plus ou moins $\arccos(\frac{X^2+Y^2-a_1^2-a_2^2}{2 a_1 a_2})$.

\section{Trajectoire du robot}

\subsection{Expliquer ce principe}



\subsection{Expliquer pourquoi il est nécessaire de choisir des vitesses
et accélérations initiales et finales nulles.}

\subsection{Donner l'algorithme qui permet de déterminer les
coefficients des polynômes d'ordre 5}

\end{document}